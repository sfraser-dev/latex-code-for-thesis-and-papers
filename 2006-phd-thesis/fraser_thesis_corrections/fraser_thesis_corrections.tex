\documentclass[a4paper,10pt]{article}
\pagestyle{plain}
\pagenumbering{arabic}

\usepackage[usenames]{color}
 
\setlength{\topmargin}{0cm}
\setlength{\textheight}{23cm}
\setlength{\textwidth}{16cm}
\setlength{\oddsidemargin}{0cm}
\setlength{\evensidemargin}{0cm}

\begin{document}

\begin{center}
\large
{\bf 
	\noindent Minor changes for \\ 
	``Multiresolutional techniques for digital image filtering and watermarking'' \\
	Stewart Ian Fraser \\ Ph.D. candidate, School of Physical Sciences, University of Aberdeen. 
}
\end{center}
\normalsize


\rule{0cm}{0.25cm}\\
Page, figure and table numbers in {\color{Blue}blue} refer to the old thesis (without minor corrections). 
Page, figure and table numbers in
{\color{Red}red} refer to the new thesis (with minor corrctions).\\
\rule{0cm}{0.25cm}\\
\noindent{\bf Global changes}
\begin{enumerate}
\item Images, such as those on p.86 and p.87, could be wider to show more detail (they are not the full textwidth).

{\color{Blue}
\underline{\noindent Changes:} \\
I have increased the width of the images on pages 
22 ({\color{Red}21}), 42 ({\color{Red}41}), 72 ({\color{Red}74}), 
85 ({\color{Red}87}), 86 ({\color{Red}88}), 87 ({\color{Red}89}), 
91 ({\color{Red}93}), 92 ({\color{Red}94}), 93 ({\color{Red}95}), 
95 ({\color{Red}97}), 96 ({\color{Red}98}), 97 ({\color{Red}99}), 
99 ({\color{Red}101}), 100 ({\color{Red}102}), 101 ({\color{Red}103}), 
106 ({\color{Red}109}), 107 ({\color{Red}110}), 133 ({\color{Red}145})
140 ({\color{Red}151}), 166 ({\color{Red}179}), 176 ({\color{Red}190}), 
242 ({\color{Red}218}), 243 ({\color{Red}219}), 246 ({\color{Red}222}), 
248 ({\color{Red}111}), 249 ({\color{Red}112}), 250 ({\color{Red}113}), 
251 ({\color{Red}114}), 252 ({\color{Red}115}), 253 ({\color{Red}116}), 
254 ({\color{Red}117}), 255 ({\color{Red}118}) 
so that they are slightly wider than the width of the text. 
Similarly, I have increased the width of the graphs on pages 169 ({\color{Red}182}), 
170 ({\color{Red}183}), 178 ({\color{Red}192}), 188 ({\color{Red}255}), 189 ({\color{Red}256}), 
202 ({\color{Red}269}), 203 ({\color{Red}270}), 207 ({\color{Red}274}), 219 ({\color{Red}282}), 
220 ({\color{Red}283}), 223 ({\color{Red}286}), 226 ({\color{Red}289}).
}

\item Figure and table captions need to convey more information 
so that the reader does not have to search through the text to find out what is going on. 

{\color{Blue}
\underline{\noindent Changes:} \\
Various captions on figures and tables have been expanded to give more information. 
These changes can be seen clearly by looking at the \emph{List of Figures} and \emph{List of Tables} 
(on pages \emph{xi} ({\color{Red}\emph{viii}}) and \emph{xv} ({\color{Red}\emph{xi}}), respectively). 
\begin{itemize}
\item Figures captions altered: 
{\color{Red}
{\color{Blue}2.1} (2.1), {\color{Blue}2.2} (2.2), {\color{Blue}2.3} (2.3), {\color{Blue}2.4} (2.4), {\color{Blue}2.5} (2.5), 
{\color{Blue}2.12} (2.12), {\color{Blue}3.4} (3.4), {\color{Blue}4.2} (4.2), {\color{Blue}4.3} (4.3), 
{\color{Blue}7.2} (7.2), {\color{Blue}7.3} (7.3), {\color{Blue}7.7} (7.7), {\color{Blue}7.10} (7.10), 
{\color{Blue}9.3} (9.3), {\color{Blue}9.4} (9.4), {\color{Blue}9.5} (9.5), {\color{Blue}9.6} (9.6), 
{\color{Blue}9.7} (9.7), {\color{Blue}9.8} (9.8), {\color{Blue}A2} (A2), {\color{Blue}A3} (A3), {\color{Blue}A4} (A4), 
{\color{Blue}C1} (B1), {\color{Blue}C2} (B2). 
}
\item Tables captions altered: 
3.1 ({\color{Red}3.1}), 5.1 ({\color{Red}5.1}), 5.2 ({\color{Red}5.2}), 5.3 ({\color{Red}5.3}), 5.4 ({\color{Red}5.4}), 
7.1 ({\color{Red}7.1}), 7.2 ({\color{Red}7.3}), 7.3 ({\color{Red}7.4}), 7.4 ({\color{Red}7.5}), 7.5 ({\color{Red}7.6}), 
7.6 ({\color{Red}7.7}), 9.1 ({\color{Red}9.1}), 9.2 ({\color{Red}9.2}), 9.3 ({\color{Red}9.3}), 9.5 ({\color{Red}9.5}), 
9.6 ({\color{Red}9.6}), 9.7 ({\color{Red}9.7}), 9.8 ({\color{Red}9.8}), 10.2 ({\color{Red}H.1}), 10.3 ({\color{Red}H.2}), 
10.4 ({\color{Red}H.3}), 10.5 ({\color{Red}H.4}), 10.6 ({\color{Red}H.5}),
10.7 ({\color{Red}H.6}), 10.8 ({\color{Red}H.7}), 10.9 ({\color{Red}H.8}), 10.10 ({\color{Red}H.9}), 
10.11 ({\color{Red}H.10}), 10.12 ({\color{Red}H.11}), 10.13 ({\color{Red}H.12}), 10.14 ({\color{Red}H.13}), 
11.1 ({\color{Red}H.14}), 11.2 ({\color{Red}H.15}), 11.3 ({\color{Red}H.16}), 11.4 ({\color{Red}H.17}), 
11.5 ({\color{Red}H.18}), 11.6 ({\color{Red}H.19}), 11.7 ({\color{Red}H.20}), 11.8 ({\color{Red}H.21}), 
11.9 ({\color{Red}H.22}).
\end{itemize}
}

\item Avoid orphans (single words or lines at the top of the page) if possible. \\
{\color{Blue}\underline{Changes:}\\
There are no more orphans.
}

\item Look at the possibility of using larger margins (not essential). \\
{\color{Blue}
\underline{Changes:} \\
Kept margins the same size, but increased the width of images and graphs (as mentioned above).
}

\item Glossary. Are all of these abbreviations necessary? e.g. Gaus. for Gaussian and FP for filter performance etc.? 
Again, infrequently used abbreviations can be hard work for the reader. \\
{\color{Blue}
\underline{\noindent Changes:} \\
The following abbreviations are no longer in the thesis (expanded versions of these abbreviations are used instead):
Av. filt (average filter), CLB (configurable logic block), 
DEC (decoded), Gaus. (Gaussian), IP (intellectual property), 
LUT (look up table), SSKF (symmetric short kernel filter), UNC (uncoded). \\
The following abbreviations are now used in equations only (in
Section {\color{Red}4.4.2}, p.{\color{Red}80}), their use in the rest of the 
thesis (\emph{e.g.} Section {\color{Red}5.1.1}\footnote{main section discussing results of denoising filters}, 
p.{\color{Red}86}),
is now extinct (full text used instead):
ES (edge sharpness), FP (filter performance), SP (this \emph{speckle reduction} abbreviation 
has been replaced by the more informative $NR_{h}$ and $NR_{e}$ 
(noise reduction homogeneous and noise reduction edge)).
}

\end{enumerate}



\noindent{\bf Structural change} \\
\indent Shorten and combine the final chapters on different error coding strategies, 
detailed results can be put into appendices if desired. As a rough guide, the page count 
for this section should be similar to those of the noise filter and new watermarking technique. \\
{\color{Blue} \underline{Changes:}\\
Old Chapters 10 and 11 combined into a single summarising chapter (Chapter {\color{Red}10}, 
p.{\color{Red}196}) which is only 9 pages long. 
This new short chapter drives home the main findings of the detailed results and analysis 
which have now been shifted into the appendix (Appendix {\color{Red}H}, p.{\color{Red}252}). 
}



\noindent
\rule{0cm}{0.2cm} \\
{\bf Questions/changes} \\
\noindent \emph{Chapter 1} 
\begin{enumerate}
\item Abbreviations not spelt out first occurrence, e.g. p.3 ECC, DCT, BCH. \\
{\color{Blue} \underline{Changes:} \\
ECC, DCT and BCH were expanded in the Abstract (this is why, initially, I did not expand them on p.3,
but I have now expanded them on p.{\color{Red}3} too). I have double checked the rest of my abbreviations and made
sure that they are fully spelt out on their first occurrence. 
}
\end{enumerate}

\noindent \emph{Chapter 2} 
\begin{enumerate}
\item p.8 ``speckle is not noise''. Untrue. Clarify that noise is any unwanted signal, regardless of whether or not it is stochastic or stationary. \\
{\color{Blue} \underline{Changes:}\\
Previous sentence of: \\ 
\emph{This interference results in ``blobs'' of different shape,
size and position to appear on the generated image, i.e., the image has been corrupted by speckle}. \\
has been replaced with (p.{\color{red}7}):\\
\emph{This unwanted interference causes ``blobs'' of different shape,
size and position to appear on the generated image, i.e., the image has been corrupted by speckle noise}.\\
The paragraph proceeding this sentence (arguing that speckle is not noise) has been removed.
}


\item p.9 Unclear argument here. 
Make it clearer that homomorphic method (logarithm) is not the implementation of the linear approximation described before it. \\
{\color{Blue} \underline{Changes:} \\
The paragraphs (p.{\color{Red}8}) describing the linear approximation method and the homomorphic method have been reworded 
to make it clear that these
two methods are seperate from each other.
}
\item p. 28 reference [34] ``Kivanc'' should be replaced by ``Mihcak''. \\
{\color{Blue} \underline{Changes:} \\
Kivanc replaced by Mih\c{c}ak ([34], p.{\color{Red}294}).
}
\item p. 30 Regarding thresholding methods, suggest inclusion of P. Moulin and J. Liu, 
``Analysis of Multiresolution Image Denoising Schemes Using Generalized Gaussian and Complexity Priors''
\emph{IEEE Trans. Info. Theory} 1999;45(3):909--919. 
The link to image statistics in transform domain (wavelets in particular) is an essential part of this analysis. \\
{\color{Blue} \underline{Changes:} \\
Moulin and Liu reference has been added. Paragraph describing the link between wavelet shrinkage denoising
and MAP estimates added to Section {\color{Red}2.3.1}, p.{\color{Red}28}.
}
\end{enumerate}

\newpage

\noindent \emph{Chapter 3}
\begin{enumerate}
\item p. 44 Regarding the issue of noise statistics in the undecimated transform domain. 
It is mentioned in Table 3.1, however the problem of correlated noise in this domain should be
discussed further. All thresholding techniques (c.f. Moulin \& Liu) are developed on the assumption of 
i.i.d. Generalized Gaussian image statistics, and additive i.i.d. Gaussian noise using MAP estimation. 
However, that does not hold in this case. See later p.53: your main idea to cope with the correlated noise 
is to have a spatially varying threshold value. Partially it is a valid assumption, just not completely justified. 
The correlated structure of noise in the undecimated domain assumes that the noise samples in some 
neighborhood will be correlated, and it suggests proceeding with window-based processing rather than 
simple pixel-wise one to capture these relationships. \\
{\color{Blue}\underline{Changes:}\\
Section {\color{Red}3.3}, p.{\color{Red}61}, renamed \emph{Summary and Future Work}. It is pointed out 
on p.{\color{Red}62} that future work
should focus upon a window based method for ascertaining the noise terms in an \emph{\`a trous} wavelet level
rather than the current pixel-wise method (as the window based method is better).
}

\item p. 63 ``Soft thresholding wavelet coefficients is known to provide....'' Usage of hard or soft thresholding 
depending on their smoothing/preserving properties as suggested in [46] is not a well-justified theoretical argument. 
In fact, according to Moulin \& Liu, the hard and soft thresholdings are the particular cases of the MAP estimate 
for various stochastic image priors. \\
{\color{Blue}\underline{Changes:}\\
Section {\color{Red}3.3}, p.{\color{Red}61}, renamed \emph{Summary and Future Work}. 
On p.{\color{Red}62}, it is stated that hard and soft thresholding
are particular cases of MAP estimates for stochastic image priors and that future work should consider these
cases.
}
\end{enumerate}

\noindent \emph{Chapter 4} 
\begin{enumerate}
\item p.68 Describe how the Watson metric is calculated. \\
{\color{Blue} \underline{Changes:} \\
New section added (Section {\color{Red}4.1.3}, p.{\color{Red}68}) 
which describes the how the Watson metric components of contrast sensitivity, luminance masking, contrast masking
and error pooling are calculted mathematically.
}

\item The fair comparisons later on depend critically on the fairness of this perceptual quality measure. 
You should include a little more text and references to support its use in this role. \\
{\color{Blue} \underline{Changes:} \\
Extra references, which utilised the Watson metric, now included (Section {\color{Red}4.1.2}, {\color{Red}p.66}).  
}
\end{enumerate}

\noindent \emph{Chapter 5 (and onwards)}
\begin{enumerate}
\item p.85 Table 5.1 � would be useful to have values for when no filtration is used (likewise table 5.4, p.99). \\
{\color{Blue} \underline{Changes:} \\
Tables {\color{Red}5.1}, {\color{Red}5.2}, {\color{Red}5.3} 
and {\color{Red}5.4} (pages {\color{Red}87}, {\color{Red}93}, {\color{Red}97} and {\color{Red}101}) 
now have an extra row describing the results for the unfiltered images.
}

\item p.104 � What did you learn from the clinical evaluation? (basically people preferred what they were used to, 
they preferred the new technique as it changed the image appearance least, which is a biased result). \\
{\color{Blue} \underline{Changes:} \\
An extra paragraph added to Section {\color{Red}5.2.2}, 
p.{\color{Red}107}, describing why saying the wavelet filter is the most preferred is a biased result.
To the final paragraph, an extra few lines have been added (p.{\color{Red}108}) saying 
that a more rigorous visual qualitative evaluation procedure could have been used (including a brief description
of how to do this). \\
\underline{Extra changes:} \\
Denoised ultrasound images have been moved from Appendix B, p.247, 
to the main body of the thesis (Section {\color{Red}5.2}, p.{\color{Red}111}).
}

\newpage

\item p.130 Table 7.1 etc. No range or standard deviation given to indicate distribution and back up claim 
about outlier on p.129 (reason for $\sigma^{2}=375$ rather than 600). \\
{\color{Blue} \underline{Changes:}\\
Ranges (minimums and maximums) and standard deviations added 
to Tables {\color{Red}7.1}, {\color{Red}7.2}, {\color{Red}7.3}, 
{\color{Red}7.5}, and {\color{Red}7.6} (p.{\color{Red}141} onwards). Table {\color{Red}7.2} is new and details
the results for a Gaussian noise attack of $\sigma^{2}=600$ which, along with new text added in 
Section {\color{Red}7.6.2} (p.{\color{Red}142}), explain the outlier result.
}


\item p.134 List number of failures in Table 7.2 as in 7.1. \\
{\color{Blue} \underline{Changes:}\\
Rows reporting the number of failures added to Tables {\color{Red}7.2}, {\color{Red}7.3}, {\color{Red}7.5} 
and {\color{Red}7.6}.
}

\item p.146 (chapter 8 onwards) remove \underline{\%} from JPEG quality factor. \\
{\color{Blue}
\underline{\noindent Changes:} \\
All the \% signs associated with JPEG quality factors have been removed.
}

\item p.133 etc. Watermark attack described as ``cropping'' does not attempt to desynchronise the
watermark, just reduce its energy (since the remaining pixels are in their original location). 
Likewise the ``scaling'' does not change the image size to desynchronise the watermark but merely 
induces error from the downscale and nearest-neighbour interpolation. The targeted scenario is 
robust watermarking. Real watermarking attacks include both signal processing attacks 
(e.g. compression, filtering, etc.) aiming at the removal watermark or it energy decrease, and geometrical attacks, 
which aim to desynchronise the watermark (c.f. Stirmark or Checkmark). 
Only signal processing attacks have been addressed here. However, the goal of attacker is to fool the 
detector (decoder) by introducing desynchronization. The task of decoder is to estimate what has happen to the 
watermarked image and to apply the resynchronization accordingly. To avoid all possible criticism you should state 
clearly somewhere that geometrical benchmarking was not your primarily goal and that you are assuming the 
presence of some synchronization mechanism (provide suitable references). \\
{\color{Blue} \underline{Changes:} \\
New section added (Section {\color{Red}6.4}, p.{\color{Red}124}) which distinguishes between signal processing attacks and 
geometrical attacks. A list of references (p.{\color{Red}125}) has been added pointing to articles 
that describe techniques for
decoder resynchronisation after geometrical attack.
In Section {\color{Red}7.6.2},
new text added (pages {\color{Red}141} and {\color{Red}142}) pointing out that the cropping attacks and half sizing attacks
performed in this chapter are not geometrical attacks and that some sort of decoder resynchronization
procedure is assumed. In the rest of this chapter, cropping is refered to as ``cropping'' and half sizing
is referred to as ``half sizing''.
}

\item p.165 Table 9.2 headings confusing, mean and std should be combined with JPEG quality range
(i.e. quality should span 3 columns, cf. table 7.1, p.130). Also table 9.5 on p.174. \\
{\color{Blue}
\underline{\noindent Changes:} \\
Tables where quality made to span three columns (as suggested): {\color{Red}9.2}, {\color{Red}9.5}, 
{\color{Red}H.2}, {\color{Red}H.8}, {\color{Red}H.10}, {\color{Red}H.12}, {\color{Red}H.13}, {\color{Red}H.14},
{\color{Red}H.16}, {\color{Red}H.17}, {\color{Red}H.19}, {\color{Red}H.20}, {\color{Red}H.22}. \\
\underline{Extra changes:} \\
In tables 10.4 ({\color{Red}H.3}), 10.10 ({\color{Red}H.9}) and 10.12 ({\color{Red}H.11}), 
multi-column \emph{ROC Area} moved from mid-table to top of table.\\
In tables 9.2 ({\color{Red}9.2}), 9.5 ({\color{Red}9.5}), 10.3 ({\color{Red}H.2}), 10.9 ({\color{Red}H.8}), 
10.11 ({\color{Red}H.10}), 10.13 ({\color{Red}H.12}), 10.14 ({\color{Red}H.13}), 11.1 ({\color{Red}H.14}), 
11.3 ({\color{Red}H.16}), 11.4 ({\color{Red}H.17}), 11.6 ({\color{Red}H.19}), 11.7 ({\color{Red}H.20}), 
11.9 ({\color{Red}H.22}), \emph{Std.} now used 
as abbreviation for \emph{standard deviation}.\\
In tables {\color{Red}7.1}, {\color{Red}7.3}, {\color{Red}7.4}, {\color{Red}7.6} 
and {\color{Red}7.7}, \emph{QF} now used as the abbreviation for \emph{Quality Factor} (to match rest of thesis).
}


\item p.171 Explain/define PIM. \\ 
{\color{Blue}
\underline{\noindent Changes:} \\
PIM overview expanded from a rough summary (in a few lines) to a more
detailed explaination (a paragraph in length including a mathematical description of PIM), p.{\color{Red}184}.
}

\newpage

\item p.196 underlining far too subtle (try bold font instead). \\
{\color{Blue}
\underline{\noindent Changes:} \\
Underlined values now changed to red colour, p.{\color{Red}263}.
}

\item Reword ``drastically'' (cannot remember where it occurred!). \\
{\color{Blue}
\underline{\noindent Changes:} \\
Bottom of Page 102 ({\color{Red}104}), \emph{drastically} removed.\\
Top of page 151 ({\color{Red}164}), \emph{drastically} changed to \emph{quite considerably}.\\
Top of page 231 ({\color{Red}207}), \emph{drastically} removed.
}
\end{enumerate}


\noindent {\bf Viva examiners, 4th November 2005}
\begin{itemize}
\item Dr. Keith A. Goatman, University of Aberdeen. 
\item Dr. Sviatoslav Voloshynovskiy, University of Geneva.
\end{itemize}

\end{document}

\newpage
This thesis investigates two interesting applications of wavelet techniques, 
namely image denoising of images containing speckle noise, and digital watermarking. 
It is generally written to a high standard and is well presented. 
The structure is good except for the final few chapters (see below). 

A minor modification of an \`a trous WT filter is presented, using an iterative technique 
(very similar to existing methods) to refine the noise estimation for each level, and a 
combined hard and soft thresholding scheme. The performance of the filter is demonstrated on the 
standard lena image and some clinical and phantom ultrasound images. 
There is some confusion about whether speckle is noise and whether or not it is useful to the 
clinician interpreting the image. 

A new watermarking technique is outlined, resulting from the 
combination of strands from two existing techniques. However, there is limited development and 
analysis of the technique. 

Finally there is a lengthy discussion about how watermarking 
robustness can be compared fairly, i.e. by fixing perceptual error and/or false positive rate. 
This discussion is however too long-winded and unfocused, which makes the thesis unbalanced. 
I recommend it be reduced from current four chapters (87 pages) to one chapter, removing a lot of 
repetition and substantially reducing the length of the thesis. 
For all its length it concentrated on a narrow range of images, and would have perhaps 
benefited from selecting/generating images to demonstrate best/worst case performance of the system. 

Overall the thesis makes the necessary distinct contribution to knowledge, shows evidence of 
originality and critical analysis. I recommend this merits a pass, subject to the 
oral examination and minor corrections.

\end{document}
