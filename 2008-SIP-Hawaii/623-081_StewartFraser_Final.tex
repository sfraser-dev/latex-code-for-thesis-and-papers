\documentclass[10pt,twocolumn,letterpaper]{article}
\usepackage{iasted}
\usepackage{floatflt}
\usepackage{amsfonts}
%\usepackage[pdftex]{graphicx,color}
\usepackage[dvips]{graphicx}

\usepackage{times}

%\setlength{\intextsep}{0.2cm}
%\setlength{\dbltextfloatsep}{-0.5cm}
%\setlength{\textfloatsep}{-0.5cm}
%\setlength{\belowcaptionskip}{-0.5cm}

\begin{document}

\date{}

%\title{AN IMPROVED REVERSIBLE WATERMARKING ALGORITHM APPLIED TO MEDICAL IMAGES}
%\title{A HIGH CAPACITY REVERSIBLE WATERMARKING ALGORITHM FOR DIGITAL IMAGES}
\title{A HIGH CAPACITY REVERSIBLE WATERMARKING TECHNIQUE BASED ON DIFFERENCE EXPANSION}

\author{Stewart I. Fraser \\
        Department of Engineering \\
        University of Aberdeen \\
        Aberdeen, AB24 3UE, UK \\
        Email: s.i.fraser@abdn.ac.uk 
        \and
        Alastair R. Allen \\
        Department of Engineering \\
        University of Aberdeen \\
        Aberdeen, AB24 3UE, UK \\
        Email: a.allen@abdn.ac.uk
}

\addtolength{\parskip}{0.4cm}

\maketitle
\thispagestyle{empty}

%%%% Replace with your abstract.
\noindent
{\bf\normalsize ABSTRACT}\newline {
Reversible image watermarking is a special case of watermarking that can embed/recover 
a user payload {\bf and} restore the original un-watermarked image {\bf exactly}.
Being able to restore the original image is important for modalities such as medical and military imaging.
This paper presents a high capacity reversible watermarking scheme based on difference
expansion. The novel method presented here improves upon a previous scheme by introducing the
companding technique. Experiments compare the novel and previous schemes using test, medical and military images.
It can be seen from the results that the novel scheme has a significantly higher payload hiding capacity than the previous scheme.


\vspace{2ex}


%%%% Replace with your keywords. 
\noindent
{\bf\normalsize KEY WORDS}\newline
{Digital Image Processing, Reversible Watermarking, Medical and Military Imaging.}

\section{Introduction}
In digital image watermarking, a hardly noticeable noise-like signal is embedded into an image to: 
\emph{(i)} protect it, 
\emph{(ii)} authenticate it or 
\emph{(iii)} enrich its information content~\cite{alattar04}. 
The watermarking process usually introduces a degradation (via bit replacement, quantization, \emph{etc}.)
into the image. 
Although this degradation is slight, it is unacceptable in certain applications which have
sensitive imagery. Two such applications are medical imaging and military imaging. 
In medical imaging, access to the unaltered original image is of paramount importance due to legal reasons~\cite{frid02}.
In military imaging, images are viewed under non-standard condtions (\emph{e.g.} extreme enhancement and zoom) thus adding a noise
like signal in the form of a watermark is unacceptable.

Reversible watermarking is a special case of watermarking with an intriguing feature:
once the watermark has been authenticated, it is possible to remove the watermark and recover the 
the original un-watermarked image {\bf exactly}~\cite{tian03}. This reversal process is achieveable
without any reference to information beyond what is in the watermarked image itself.
Various methods for reversible watermarking have been proposed, such as schemes that use 
data compression~\cite{frid02, celik05}, schemes that use histogram bin exchanging~\cite{chang05, vlees03, yang04} and
schemes that use difference expansion~\cite{alattar04, tian03, weng07, thodi04}.
%Rev WM \cite{tian03, weng07, feng06, alattar04, celik05, chang05, frid02, thodi04, vlees03, yang04}
%Rasta order. Insert twice.
In this paper, a new high capacity reversible watermarking method based on difference expansion is introduced.
The performance of this new method is assessed
using test, medical and military images.

The novel reversible watermarking method described here is based upon the Tian method~\cite{tian03}; 
%The novel method
%improves upon the Tian method by applying a companding technique. 
it improves upon the Tian method by introducing a companding technique. 
This allows the novel scheme to embed a significantly larger payload than is possible with the Tian scheme.
%Results comparing the novel and the Tian methods using test, medical and military images verify the superior 
%performance of the novel method.
%which allows a significantly larger payload to be embedded into an image
%than is possible with the Tian method. 
Like the Tian method, the novel method can recover the original image {\bf exactly}.
%(see Section~\ref{sec:compand}).
%The novel method (using the companding technique) allows a significantly larger payload to be embedded into the image
%than is possible with the Tian method. Like the Tian method, the new method can recover the original image {\bf exactly}.

%In Section~\ref{sec:tian}, an overview of the Tian reversible watermarking method is given.
%In Section~\ref{sec:novel}, the novel reversible watermarking method is presented and improvements over the
%Tian method are highlighted.
%Results comparing the novel and the Tian methods using test, medical and military images are given in Section~\ref{sec:results}.
%Conclusions are drawn in Section~\ref{sec:conc}.


\section{Overview of the Tian method}
\label{sec:tian}
In \cite{tian03}, Tian introduced a reversible watermaring algorithm based upon difference expansion of adjacent
pixels. The difference between adjacent pixels was calculated via an integer transform.  

\subsection{An integer transform}
\label{sec:intTx}
%In the Tian method, a watermark is embedded in the difference between neighbouring pixel values.
For a pair of pixel values $(x,y)$ in a grayscale image \emph{I} ($x,y \in \mathbb{Z}$;
$0 \leq x,y \leq 255$), their average value $l$ is defined as:
\begin{equation}
\label{eq:l}
l = \left \lfloor \frac{x+y}{2} \right \rfloor 
\end{equation}
and thier difference value $h$ is defined as:
\begin{equation}
\label{eq:h}
h=x-y
\end{equation}
where $\lfloor \cdot \rfloor$ is the floor function.
The original $x$ and $y$ values can be obtained from $l$ and $h$
by applying the inverse integer transform described in
equations~\ref{eq:x} and \ref{eq:y}:
\begin{equation}
\label{eq:x}
x = l + \left \lfloor \frac{h+1}{2} \right \rfloor
\end{equation}
\begin{equation}
\label{eq:y}
y=l - \left \lfloor \frac{h}{2} \right \rfloor
\end{equation}

\subsection{Watermark embedding via difference expansion}
In \cite{tian03}, watermark information is embedded in a digital image by altering the 
difference value $h$. There are certain restrictions in the alteration of the $h$ value
that must be adhered to.
As grayscale values ($x,y$) in digital images are bounded in the range $[0,255]$,
the inverse integer transform must meet the following two condtions:
\begin{equation}
0 \leq l + \left \lfloor \frac{h+1}{2} \right \rfloor \leq 255
\end{equation}
and
\begin{equation}
0 \leq l - \left \lfloor \frac{h}{2} \right \rfloor \leq 255
\end{equation}
which is equivalent to:
\begin{equation}
\label{eq:hCon}
|h| \leq \mbox{min} (2(255-l), 2l+1)
\end{equation}
Therefore, to prevent underflow and overflow occurring in the watermarked image, 
the difference value $h$ must satisfy equation~\ref{eq:hCon}. 
Adhering to equation~\ref{eq:hCon}, the $h$ value is altered to embed a 
watermark bit. There are two methods in which to alter $h$:
\begin{enumerate} 
	\item {\bf Change} the difference value $h$
	\item {\bf Expand} the difference value $h$
\end{enumerate} 
If a neighbouring pixel pair cannot be changed or expanded, they are left unaltered.

\subsubsection{Changing the difference value $h$}
\label{sec:change}
For a neighbouring grayscale pair of pixels ($x,y$), the difference value
$h$ is changeable only if:
\begin{equation}
\label{eq:chFind}
\left | \left \lfloor \frac{h}{2} \right \rfloor \cdot 2 + b \right | \leq \mbox{min}(2(255-l),2l+1)
\end{equation}
for both $b=0$ and $b=1$ ($b$ is a watermark bit).
Changing $h$ requires removing its LSB (Least Significant Bit) and replacing it with a watermark bit:
\begin{equation}
\label{eq:ch}
h_{new} = \left \lfloor \frac{h}{2} \right \rfloor \cdot 2 + b
\end{equation}
The removed LSB from $h$ must be stored in order to 
recover the original image exactly after watermark extraction. 
An exception is made if $h$ is 1 or -2, no LSB is stored.
Note that if a pixel pair are identified as changeable and $h$ has a value of 0
or -1, then the pair are left unaltered\footnote{this is because the conditions
for changeable and expandable are the same when $h$ is 0 or -1; expanding  
is preferred over changing}.
Thus, changing the value of $h$ does
not provide extra storage space.

\subsubsection{Expanding the difference value $h$}
\label{sec:ex}
On the other hand, expanding $h$ does provide extra storage space.
Neighbouring grayscale pairs of pixels ($x,y$) are expandable only if:
\begin{equation}
\label{eq:exFind}
\left | 2 \cdot h + b \right | \leq \mbox{min}(2(255-l),2l+1)
\end{equation}
for both $b=0$ and $b=1$ ($b$ is a watermark bit).
Expanding $h$ requires left shifting all the bits by one and using the
freed LSB to store the watermark bit:
\begin{equation}
\label{eq:ex}
h_{new} = 2 \cdot h + b
\end{equation}
Thus, if a pair of neighbouring pixels are expandable, one extra bit of storage
space is obtained. This is the concept of difference expansion.

It is important to note that an expandable $h$ is also changeable. After embedding
a watermark bit via difference expansion, an expanded $h_{new}$ is still changable.
This relationship is exploited in watermark recovery.

In order to reduce the visual degradation to a marked image, a user defined threshold is introduced to
expandable pairs. If $h_{new}$ in an expandable pair is greater than or equal to this threshold,
the pair are changed rather than expanded, thus minimizing the Mean Squared Error (MSE) of the watermarked image.
However, this threshold causes a detection problem~\cite{feng06}. In some situations, the 
detector cannot tell if a pair has been changed or expanded. For example, suppose that the threshold
value is $10$, the difference value $h = 6$ and the watermark 
value to be embedded is $1$. From equation~\ref{eq:ex}, $h_{new} = 2 \cdot h + b = 2 \cdot 6 + 1 = 13$. 
In the retrieving phase at the detector, there are now {\bf two} possibilities: $h$ was $6$ (embedded)
or $h$ was 13 (not embedded, larger than or equal to threshold). In order to combat this problem, it is necessary
to create a binary location map marking the position of all expanded pairs.

\subsection{Constructing the watermark bitstream}
\label{sec:construct}
In Tian's reversible watermarking scheme, 
the watermark bitstream {\bf W} that is embedded has three distinct components.
These are: 
\begin{itemize}
\item {\bf L}: a binary location map marking the location of expandable pairs.
The location map is losslessly compressed\footnote{JBIG lossless compression
is used in this paper} via JBIG compression~\cite{kim97}
or run-length coding.
The greater the compression of {\bf L}, the greater the space there will 
be for the payload {\bf P}.
An end of message symbol is appended to the end of {\bf L}.
\item {\bf C}: a vector containing all the removed LSBs from the changeable pairs.
\item {\bf P}: the payload the user wishes to embed.
\end{itemize}
These vectors are concatenated together to produce a watermark bitstream 
$\mbox{{\bf W}} = \mbox{{\bf L}} \cup \mbox{{\bf C}} \cup \mbox{{\bf P}}$.
Thus {\bf W} contains both the user inserted payload ({\bf P}) and the information
required ({\bf L} and {\bf C}) to exactly restore the original image. This is shown
diagramatically in Figure~\ref{fig:origLCP}.
\begin{figure}[!htb]
%\setlength{\abovecaptionskip}{-0.25cm}
\centerline{ \hbox{
        %\includegraphics[height=7cm,width=8cm]{graphsPics/tianComp4.pstex}
        \includegraphics[height=1cm,width=7cm]{graphsPics/originalLCP.pstex}
}}
%	\rule{2cm}{0cm}
        \caption{Watermark structure of the Tian scheme}
        \label{fig:origLCP}
\setlength{\abovecaptionskip}{0cm}
\end{figure}

\subsection{Watermark recovery}
\label{sec:wmrec}
The integer transform (described in section~\ref{sec:intTx}) is applied to a watermarked image \emph{I'}.
Using equation~\ref{eq:chFind}, changeable and expandable pairs are found ({\bf CE'}). 
Note that, by definition, expandable pairs are also changeable, thus equation~\ref{eq:chFind}
finds both changed and expanded pairs.

It can be seen from equations~\ref{eq:ch}
and \ref{eq:ex} that the original watermark bitstream {\bf W} was inserted into the LSBs of changeable and
expandable difference values. Thus, collecting the LSBs from the difference values of {\bf CE'} will 
result in the recovered watermark bitstream {\bf W'}. 

By identifying the end of message symbol in {\bf W'}, the 
bits from the start until the end of message symbol are losslessly decompressed
(using a JBIG decoder) to retrieve the binary location map.
This location map gives the locations of all expanded pairs.

\subsubsection{Restoring the original image exactly}
\label{sec:restore}
The bits in {\bf W'} after the location map {\bf L'} are the stored LSBs of changeable pairs {\bf C'}. 
Using equations~\ref{eq:l} and \ref{eq:h}, the integer transform is applied to the watermarked image \emph{I'} to obtain
$l'$ and $h'$. Using the stored LSBs ($c_{i}$) from {\bf C'}
and the location map values, the original difference values $h_{o}$ can be computed. For each pixel pair:
\begin{itemize}
	\item If pixel pair are non changeable, do nothing \\ ({\bf unaltered pair})
	\item If pixel pair are changeable and location map is 1 \\ ({\bf expandable pair}) \\ \rule{0.2cm}{0cm} $h_{o} = \lfloor h'/2 \rfloor$
	\item If pixel pair are changeable and location map is 0 \\ ({\bf changeable pair}) 
	\begin{itemize}
		\item if ($h'==0$) or ($h'==1$) \\ \rule{0.2cm}{0cm} $h_{o}=1$;
		\item else if ($h'==-1$) or ($h'==-2$) \\ \rule{0.2cm}{0cm} $h_{o}=-2$;
		\item else \\ \rule{0.2cm}{0cm} $h_{o}= \lfloor h'/2 \rfloor \cdot 2 + c_{i}$; \\ \rule{0.2cm}{0cm} $i=i+1$;
	\end{itemize}
\end{itemize}
Applying the inverse integer transform, the restored image values ($x_{r}$ and $y_{r}$) 
for each pixel pair can be calculated
to obtain the original un-watermarked image {\bf exactly}:
\begin{equation}
x_{r} = l' + \left \lfloor \frac{h_{o}+1}{2} \right \rfloor 
\end{equation}
\begin{equation}
y_{r} = l' - \left \lfloor \frac{h_{o}}{2} \right \rfloor 
\end{equation}
Once all the {\bf C'} values have been used to restore the original un-watermarked image, the rest of the values in {\bf W'}
(after removal of the location map {\bf L'}) belong to the user inserted payload {\bf P'}.

\section{A novel high capacity reversible watermarking method}
\label{sec:novel}
A novel reversible watermarking method is described here that improves upon the Tian scheme. 
The novel method adapts the Tian scheme to incorporate companding
of difference values greater than or equal to a user defined threshold.

\subsection{Companding technique}
\label{sec:compand}
In~\cite{weng07}, Weng \emph{et al.} presented a reversible watermarking scheme. This scheme used the concept of 
companding. 
Companding is the process of {\bf comp}ressing a signal followed by exp{\bf anding} (decompressing\footnote{since 
the term ``expanding'' is used in this paper to describe the expansion of difference values ($h$), the exp{\bf anding} process
of companding will be referred to as decompressing}) the signal. 
Let $C$ be a compression function and let $D$ be a decompressing function. For a signal $x$, $C$ and $D$ have the following 
relationship: $D(C(x)) = x$. For a digital signal, $C_{q}$ and $D_{q}$ represent the quantized versions of $C$ and $D$, 
respectively, and $q$ denotes the quantization function. The compression function $C_{q}$ is given by:
\begin{equation}
\label{eq:comp}
 x_{q} = C_{q}(x) = \left \{ 	\begin{array}{ll}
				x & |x| < th \\
				\mbox{sgn}(x) \times \left( \left \lfloor \frac{|x| - th}{2} \right \rfloor + th \right) & |x| \geq th 
				\end{array}
			\right.
\end{equation}
where $\mbox{sgn}(\cdot)$ is the sign function and $th$ is a preset threshold chosen by the user.
The decompression function $D_{q}$ is given by:
\begin{equation}
\label{eq:exp}
D_{q} = \left \{	\begin{array}{ll}
			x & |x| < th \\
			\mbox{sgn}(x) \times (2|x| - th) & |x| \geq th
			\end{array}
	\right.
\end{equation}
Companding values via equations \ref{eq:comp} and \ref{eq:exp} leads to no errors when 
$|x| < th$. However, when $|x| \geq th$, companding errors $r$ occur:
\begin{equation}
\label{eq:error}
r = |x| - |D_{q}(C_{q}(x))|
\end{equation}
where $r \in \{0,1\}$. 



\subsection{Novel method using the companding technique}
\label{sec:novelCompand}
The novel method presented in this paper treats unaltered and changeable pixel pairs the same way as the Tian method.
The novel method differs from the Tian method in the way that it treats expandable pairs; 
the novel scheme incorporates companding 
whereas the Tian scheme does not.

The Tian scheme finds expandable pairs using equation~\ref{eq:exFind}.
As described in Section~\ref{sec:ex}, if the difference value $h$ of an expandable pair is greater than 
or equal to a user defined threshold, the pair would be {\bf changed} rather than {\bf expanded}. 
This was done to minimize the MSE of a watermarked image. 
As mentioned in Section~\ref{sec:change}, changeable pairs
do not provide extra storage space; they carry one bit of watermark
information in thier LSBs
but require their original LSB values to be collected in {\bf C}
(which constitutes part of the watermark bitstream {\bf W}, as shown in Figure~\ref{fig:origLCP}).

In the novel scheme, if the difference value $h$ of an expandable pair is greater than
or equal to a user defined threshold, the $h$ value is compressed\footnote{compressed, the 
first part of companding} (via equation~\ref{eq:comp}) and then expanded\footnote{values left 
shifted and watermark bit stored in LSB}
(via equation~\ref{eq:ex}). Thus, $h$ values $\geq th$ that would have been {\bf changed} in the Tian scheme are now 
{\bf compressed} and then {\bf expanded}
in the novel scheme. 
To be able to recover the original image exactly, 
it is necessary to collect the companding errors (equation~\ref{eq:error}) into 
a vector {\bf R}. This vector is embedded into the image as part of the watermark
$\mbox{{\bf W}} = \mbox{{\bf L}} \cup \mbox{{\bf C}} \cup \mbox{{\bf R}} \cup \mbox{{\bf P}}$ (shown in Figure~\ref{fig:tianLCRP}).
It is important to note that no additional bits of storage are obtained from compressing and expanding as the companding errors
have to be stored in {\bf R}. However, having more pixel pairs expanded rather than changed makes the location map more sparse and
thus more compressible. This is elaborated upon in Section~\ref{sec:improvements}.

\begin{figure}[!htb]
%\setlength{\abovecaptionskip}{-0.25cm}
\centerline{ \hbox{
        \includegraphics[height=1cm,width=8cm]{graphsPics/tianLCRP.pstex}
}}
%	\rule{2cm}{0cm}
        \caption{Watermark structure of the novel scheme}
        \label{fig:tianLCRP}
\setlength{\abovecaptionskip}{0cm}
\end{figure}

The following MATLAB~\cite{matlab2} code snippet shows how the difference value of an expandable pixel pair 
is companded (using equations~\ref{eq:comp} and \ref{eq:exp}) and how the companding error is collected in {\bf R}.

\begin{scriptsize}
\begin{verbatim}
% blockIn is a 1x2 non-overlapping block
x=blockIn(1);           
y=blockIn(2);
% Find pixel pair average
l = floor( (x+y)/2 );   
% Find pixel pair difference value
h = x - y;              
% If the difference value is greater 
% than or equal to threshold ...
if abs(h) >= th
   % Compress h (companding process)
   hCompress = sign(h) * (floor((abs(h)-th)/2)+th);       
   % Use compressed difference value (in later stages)
   h = hCompress;
   % Decompress h (companding process)
   hDecompress= sign(hCompress) * ((2*abs(hCompress))-th);   
   % Calculate companding error
   r = abs(hDecompress - h);                                  
   % Store companding error in bitstream R
   R_bitstream=[R_bitstream r];                           
end
\end{verbatim}
\end{scriptsize}

\subsection{Watermark recovery and restoring the original image exactly}
\label{sec:restoreNovel}
The process used by Tian (described in 
Section~\ref{sec:wmrec}) to recover the watermark bitstream {\bf W'}
from the LSBs of difference values in the watermarked image, 
is also used in the novel scheme.
Once {\bf W'} has been obtained, the location map {\bf L'} can be recovered by identifying the
end of message symbol. 

The bits in {\bf W'} after the location map {\bf L'} are the stored LSBs of changeable pairs {\bf C'} and 
the companding errors {\bf R'}. 
Using equations~\ref{eq:l} and \ref{eq:h}, the integer transform is applied to the watermarked image \emph{I'} to obtain
$l'$ and $h'$. Using the stored LSBs ($c_{i}$) from {\bf C'}, the companding errors ($r_{j}$) from {\bf R}
and the location map values, the original difference values $h_{o}$ can be computed. For each pixel pair:
\begin{itemize}
	\item If pixel pair are non changeable, do nothing \\ ({\bf unaltered pair})
	\item If the pixel pair are changeable and location map is 1 \\ ({\bf expandable pair})
	\begin{itemize}
		\item $h_{o} = \lfloor h'/2 \rfloor$
		\item if ($|h_{o}| \geq th$) \\ \rule{0.2cm}{0cm} $h_{o} = \mbox{sgn}(h_{o}) \cdot ((2 \cdot \mbox{abs}(h_{o})) - th + r_{j})$;\\
			\rule{0.2cm}{0cm} $j=j+1$;
	\end{itemize}
	\item If the pixel pair are changeable and location map is 0 \\ ({\bf changeable pair}) 
	\begin{itemize}
		\item if ($h'==0$) or ($h'==1$) \\ \rule{0.2cm}{0cm} $h_{o}=1$;
		\item else if ($h'==-1$) or ($h'==-2$) \\ \rule{0.2cm}{0cm} $h_{o}=-2$;
		\item else \\ \rule{0.2cm}{0cm} $h_{o}= \lfloor h'/2 \rfloor \cdot 2 + c_{i}$; \\ \rule{0.2cm}{0cm} $i=i+1$;
	\end{itemize}
\end{itemize}
Applying the inverse integer transform, the restored image values ($x_{r}$ and $y_{r}$) 
for each pixel pair can be calculated
to obtain the original un-watermarked image {\bf exactly}:
\begin{equation}
x_{r} = l' + \left \lfloor \frac{h_{o}+1}{2} \right \rfloor 
\end{equation}
\begin{equation}
y_{r} = l' - \left \lfloor \frac{h_{o}}{2} \right \rfloor 
\end{equation}
Once all the {\bf C'} and {\bf R'} values have been used to restore the original un-watermarked image, the rest of the values in {\bf W'}
(after removal of the location map {\bf L'}) belong to the user inserted payload {\bf P'}.

\subsubsection{Improvement obtained}
\label{sec:improvements}
As stated in Section~\ref{sec:novelCompand}, using the companding technique in the novel method does not provide
extra bits of storage. The companding technique allows pixels pairs that would have been changed in
the Tian scheme to instead be expanded in the novel scheme.
This is important as it makes the location map of the novel scheme more sparse than the location map of the Tian scheme.
Sparse location maps can be compressed more than non-sparse location maps. Thus, the novel scheme achieves
higher payload capacities than the Tian scheme as it
produces location maps that are more compressible.

An example of this is given in 
Figure~\ref{fig:locmaps}, where binary location maps for the original Tian scheme (a) and the novel
scheme (b) are shown\footnote{original input image was grayscale Lena $512 \times 512$, threshold values of 16 were used in both
the Tian and novel schemes}. 
\begin{figure}[!htb]
\setlength{\abovecaptionskip}{-0.25cm}
\centerline{ \hbox{
        \includegraphics[height=7cm,width=8cm]{graphsPics/fig5.pstex}
}}
%	\rule{2cm}{0cm}
        \caption{Location maps for {\bf (a)} Tian scheme {\bf (b)} new improved scheme; grayscale Lena $512\times 512$, threshold of 16}
        \label{fig:locmaps}
\setlength{\abovecaptionskip}{0cm}
\end{figure}
Non-overlapping blocks of $1\times 2$ were used to generate the pixels pairs in these location maps, hence
the width is half the height.
In Figure~\ref{fig:locmaps},
white pixels are expandable and black pixels are not. It can be seen that Figure~\ref{fig:locmaps}(b)
is much more sparse than Figure~\ref{fig:locmaps}(a). Thus, when these location maps are losslessly compressed,
greater compression will be achieved from Figure~\ref{fig:locmaps}(b) than from Figure~\ref{fig:locmaps}(a).
The more compression that can be applied to the location map {\bf L}, the more space there is for the user payload {\bf P}.

The location maps in Figure~\ref{fig:locmaps} were stored as binary Portable BitMaps (PBM) and then losslessly compressed 
via JBIG compression; Table~\ref{tab:locmaps} shows the file sizes for each.
Clearly, the sparse location map of the novel scheme can be compressed much more 
than the location map of the Tian scheme.  
\begin{table}[!ht]
\setlength{\abovecaptionskip}{-0.25cm}
\begin{center}
\begin{tabular}{|c|c|c|} \hline
Scheme 		& Original PBM 	& JBIG compressed \\ \hline \hline
Tian method 	& 16396 Bytes	& 3785 Bytes \\ \hline 
novel method 	& 16396 Bytes	& 101 Bytes \\ \hline 
\end{tabular}
\caption{Original and compressed file sizes of location maps; grayscale Lena $512\times 512$, threshold of 16}
\label{tab:locmaps}
\end{center}
\setlength{\abovecaptionskip}{0cm}
\end{table}

\vspace{-1.25cm}
In summary, the novel scheme (using the companding technique) saves space by producing highly compressible 
location maps which in turn allows larger user payloads to be inserted.

\section{Results}
\label{sec:results}
The performance of the novel scheme is compared to that of the Tian scheme using different images.
The images used are: (\emph{i}) Lena (test image), (\emph{ii}) Eye (SLO\footnote{Scanning Laser Ophthalmoscope} medical image) 
and (\emph{iii}) Aerial (SAR\footnote{Synthetic Aperture Radar} military image). 

Initially, for both the novel and Tian schemes, 
vertical pixel pairs are selected (using non-overlapping blocks of $2\times 1$).
In order to achive payload bitrates greater than 0.5 bits per pixel (bpp), 
it is necessary to perform multiple embedding.
This consists of embedding a watermark (into image $I$) using vertical pixel pairs to obtain a watermarked image $W_{v}$.
Taking advantage of the lossless nature of reversible watermarking, another watermark is embedded
into the horizontal pairs of $W_{v}$ (using non-overlapping blocks of $1\times 2$) to obtain $W_{vh}$.
Thus, the original image $I$ has been watermarked twice resulting in $W_{vh}$.
The original image can be recovered exactly by first decoding $W_{vh}$ to get $W_{v}$ and then
decoding $W_{v}$ to get $I$.



Tables~\ref{tab:lena512}, \ref{tab:eye256} and \ref{tab:SAR256} show the results for 
the Lena, Eye and SAR images, respectively. Each table compares the performance of the Tian and novel schemes
via hiding capacity (measured using payload bitrate) and image degradation
(measured using Peak Signal to Noise Ratio (PSNR)).
Grayscale Lena $512 \times 512$ was chosen as the test image as this was the image tested in~\cite{tian03}. Thus, a fair
quantitative comparison between the results published in~\cite{tian03} and the results presented here can be obtained.

In Table~\ref{tab:lena512}, it can be seen that the novel scheme is performing better than the Tian scheme. For 
approximately equal payload bitrates, the novel scheme returns better quality (higher PSNR) watermarked images.
For example, a payload bitrate of 
0.24bpp\footnote{payload capacity = image size $\times$ bpp = $512\times 512\times 0.24 \approx 62914$ bits; the higher 
the bpp value, the greater the hiding capacity} in the Tian
scheme results in a watermarked image of 42.86dB whereas the same payload bitrate in the novel scheme results in 
a PSNR of 45.42dB.
In general, the novel scheme outperforms the Tian scheme 
%more at the lower payload bitrates and less at higher payload rates.
at every bitrate tested. The increase in performance is greater at low bitrates than it is at high bitrates.

The results in Table~\ref{tab:eye256} (for the medical image) follow a similar pattern to 
those in Table~\ref{tab:lena512}, with the novel scheme
performing better than the Tian scheme over the complete range of payload bitrates (especially at lower bitrates). 
For a given bitrate, the PSNR of the novel scheme is higher (better) than the PSNR of the Tian scheme.
It can be seen in Table~\ref{tab:SAR256} (SAR image) 
that the novel scheme is again outperfoming the Tian scheme over the whole range of payload bitrates. 
However, the results for both the novel and 
Tian schemes using the SAR image are not as impressive as the results obtained using the Lena or
Eye images. This may attributed to the fact that both schemes perform better in homogeneous
images than they do in inhomogeneous images\footnote{the SAR image is inhomogeneous due 
to speckle noise} (in general, homogeneous images will have more expandable pairs).

Figure~\ref{fig:nineFigs} shows the original Lena, Eye and SAR images. Examples of the bitrates
achieved and the corresponding PSNRs using the novel and Tian schemes are given. It can be seen that
for approximately equal measures of PSNR, the novel scheme returns much higher bitrates than the Tian 
scheme. 

%These results, using test, medical and military images, clearly show that the novel scheme can embed
%greater payload bitrates than the Tian scheme for a given PSNR.

\section{Conclusions}
\label{sec:conc}
%Future work JBIG2, arithmetic coding, more medical images
A novel reversible watermarking scheme implementing
the companding technique has been presented. This novel scheme 
adapted the Tian scheme to implement companding of certain pixel 
pairs. This resulted in location maps that were sparser in the 
novel scheme than they were in the Tian scheme. This in turn
allowed the novel scheme to have a higher payload hiding capacity than the
Tian scheme (for similar PSNR values). This was verified via
experimental results using test, medical and military images.


\begin{table*}[!p]
\begin{center}
%\vspace{-1.0cm}
\begin{tabular}{|l|c|c|c|c|c|c|c|c|c|} \hline
%\setlength{\abovecaptionskip}{-2.5cm}
Tian: payload bitrate (bpp) 	& 0.15 	& 0.24	& 0.32	& 0.39	& 0.46	& 0.54	& 0.67	& 0.74	& 0.85	\\ \hline
Tian: PSNR (dB)			& 44.20 & 42.86	& 41.55	& 40.06	& 37.66	& 36.15	& 34.80	& 33.05	& 32.54	\\ \hline \hline
novel: payload bitrate (bpp)	& 0.16 	& 0.24	& 0.37	& 0.41	& 0.48	& 0.57	& 0.70	& 0.78	& 0.87	\\ \hline
novel: PSNR (dB)		& 47.09	& 45.42	& 42.45	& 41.22	& 40.35	& 38.84	& 36.78	& 35.31	& 33.68	\\ \hline
\end{tabular}
\caption{Results for the Lena test image; grayscale $512\times 512$}
\label{tab:lena512}
\end{center}
\setlength{\abovecaptionskip}{0cm}
\end{table*}

\begin{table*}[!p]
%\setlength{\abovecaptionskip}{-0.75cm}
\begin{center}
\vspace{-0.7cm}
\begin{tabular}{|l|c|c|c|c|c|c|c|c|c|} \hline
Tian: payload bitrate (bpp) 	& 0.23 	& 0.29 	& 0.34	& 0.47	& 0.53	& 0.66	& 0.79	& 0.86	& 0.94	\\ \hline
Tian: PSNR (dB)			& 42.53 & 41.62	& 40.94	& 37.87	& 35.97	& 34.89	& 33.53	& 32.73	& 31.65	\\ \hline \hline
novel: payload bitrate (bpp)	& 0.24  & 0.29  & 0.36	& 0.43	& 0.60	& 0.66	& 0.81	& 0.89	& 0.94	\\ \hline
novel: PSNR (dB)		& 47.09	& 45.27	& 42.93	& 40.54	& 39.61	& 38.48	& 35.71	& 33.77	& 32.06	\\ \hline
\end{tabular}
\caption{Results for the Eye medical image; grayscale $256 \times 256$}
\label{tab:eye256}
\end{center}
\setlength{\abovecaptionskip}{0cm}
\end{table*}

\begin{table*}[!p]
%\setlength{\abovecaptionskip}{-0.25cm}
\begin{center}
\vspace{-0.7cm}
\begin{tabular}{|l|c|c|c|c|c|c|c|c|c|} \hline
Tian: payload bitrate (bpp) 	& 0.18 	& 0.25 	& 0.33	& 0.47	& 0.52	& 0.64	& 0.76 	& 0.85	& 0.90	\\ \hline
Tian: PSNR (dB)			& 36.30 & 35.29	& 33.94	& 29.88	& 29.50	& 28.54	& 27.46	& 26.58	& 26.02	\\ \hline \hline
novel: payload bitrate (bpp)	& 0.19 	& 0.25  & 0.33	& 0.45	& 0.54	& 0.66 	& 0.77 	& 0.85	& 0.90	\\ \hline
novel: PSNR (dB)		& 40.71	& 38.74	& 36.32	& 34.40	& 32.87	& 30.81	& 28.99	& 27.41	& 26.32	\\ \hline
\end{tabular}
\caption{Results for the SAR military image; grayscale $256 \times 256$}
\label{tab:SAR256}
\end{center}
\setlength{\abovecaptionskip}{0cm}
\end{table*}

\begin{figure*}[p]
\setlength{\abovecaptionskip}{-0.25cm}
	\vspace{-0.45cm}
\centerline{ \hbox{
        \includegraphics[height=12.75cm,width=12.75cm]{graphsPics/results/nine.pstex}
}}
        \caption{Original images and watermarked images (with approximately equal PSNRs): 
	{\bf (a)} original Lena image (zoomed) {\bf (b)} Tian: payload = 0.39bpp, PSNR = 40.06dB 
	{\bf (c)} novel: payload = 0.48bpp, PSNR = 40.35dB
	{\bf (d)} original Eye image {\bf (e)} Tian: payload = 0.47bpp, PSNR = 37.87dB {\bf (f)} novel: payload = 0.66bpp, PSNR = 38.48dB
	{\bf (g)} original SAR image {\bf (h)} Tian: payload = 0.18bpp, PSNR = 36.30dB {\bf (i)} novel: payload = 0.33bpp, PSNR = 36.32dB}
        \label{fig:nineFigs}
\setlength{\abovecaptionskip}{0cm}
\end{figure*}


\renewcommand{\baselinestretch}{0.9}
	%\footnotesize
	\small
%%%%%%%%%%%%%%%%%%%%%%%%%%%%%
%%%%%%%%%%%%%%%%%%%%%%%%%%%%%
\bibliography{hawaii08}
\bibliographystyle{unsrt}
	\normalsize
\renewcommand{\baselinestretch}{1}
%%%%%%%%%%%%%%%%%%%%%%%%%%%%%%%%%%%%%%%%%%%%%%%%%%%%%%%%%%%%%%%%%%%%%%%%%%%%%%%%%%%%%%%%%%%%%%%%%%%% 

\end{document}



